\section{Balancing Learning Group}

One point of discussion is whether the make-up of learning groups effects their
performance of a learning task. If a collection of students were assessed on
their competence on a ten point scale, it is not clear if balancing group 
ability would enhance the performance of the learners.

While there is evidence that for some learning activities like software 
engineering pair learning has produced positive results, there is less empirical
research into using pair approaches. This seems largely due to a preference for
learning by 'rote'. 

In many deprived contexts one of the major challenges for the teacher is simply
the delivery of all of the content necessary for the curriculum. It seems
though that this style of teaching is preferred due to its simplicity; due to
the fact that lessons can be planned minute by minute in advance.

As well as the assessment of students an eLearning system could provide tools to
allow group selection based on previous performance. Assuming that all grades
of achievement could be combined the system could regularly regroup students
in order to create balanced teams. The goal here being to help the students who
are 'lagging' behind others by pairing them with other students who are 
progressing more rapidly in a particular subject. 
