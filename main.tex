\documentclass[11pt]{IEEEtran}

\usepackage{graphicx}
\usepackage[style=ieee,backend=bibtex8]{biblatex}
\bibliography{bibliography}

\begin{document}

\title{Towards a Self Guided Team Based eLearning Solution}

\author{D.~K.~Allison,~J.~Harty}

\maketitle

\markboth{Towards Self Guided ELearning}{}

\begin{abstract}
Millions of learners have little chance of success in terms of personal 
development globally, the schools are missing some of the teachers, they have 
very few resources and the teaching is rote-based. eLearning has been touted as 
one way to compensate for the lack of teachers and the lack of other learning
resources, however there's patchy evidence that the currently available 
resources materially help the pupils learn skills that matter. Also the 
materials may not be particularly suited to the level, or learning syles of the 
pupils. Also the materials may be very relevant to the location, language, or 
context of the school.

This position paper captures the essence of an elearning system which may help 
to address many of the concerns, and be viable in environments and contexts were
there are few trained teachers available. It supports team-based learning where 
the learners have to be able to work unaided for extended periods. The paper 
also explores whether and how differences in the skill levels of participants 
can be used to improve the learning, and whether the pairings (etc) break down 
if there are large differences between the skill levels of the participants. In 
some cases, algorithms may be able to detect common-modes of mistakes, such as 
where learners mistakenly believe 0 - any number = 0 and pair that learner with 
someone who understands the correct approach to solving this problem so that 
pupils help each other to learn and develop their skills.

It may also be suitable and relevant for situations and contexts where there are
more, and more competent teachers, and better resources and facilities 
available. The authors are interested in designing and devising a reference
implementation available as opensource software with materials available under
permissive sharing licenses such as creative commons.
\end{abstract}

\begin{IEEEkeywords}
eLearning, paired learning, digital assessment
\end{IEEEkeywords}

% Main section includes
\section{Objectives}

\begin{itemize}
\item Support partially or unguided learning.
\item Support classification of assessment failures.
\item Support tree based learning patterns.
\item Support creation of balanced learning teams.
\item Support teacher assessment of learning needs in real time.
\end{itemize}



\section*{Related work}
Various learning packages are in widespread use, these include opensource projects such as \cite{kalite} which includes tracking progress of logged-in learners, and others such as \cite{rachel} that provide various materials using a local web server that serves opencontent such as \cite{wikipedia-for-schools} and optionally \cite{kalite}. Of the various opensource projects the one best able to support localisation is \cite{kalite}. 

The providers of \cite{kalite}, an orgainsation called \cite{learning-equality}, want to develop a new platform able to incorporate a wider range of materials and topics from many sources, rather than the pure \cite{khanacademy} contents that \cite{kalite} incorporates.

\section*{Needs}
There are various needs or requirements to enable and support learning globally. These broadly include locally curated contents and materials, learning in languages in local use by the learners, and content that can be adapted to the context of the learner, for instance with local buildings, heat sources, curriculum topics, etc. 

% Include all citations in the database
\nocite{*}


% A non numbered section
\section*{Acknowledgements}
The authors wish to thank the anonymous reviewers for their valuable
suggestions.

\printbibliography

\end{document}



