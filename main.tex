\documentclass[11pt]{IEEEtran}

\usepackage{graphicx}
\usepackage[style=ieee,backend=bibtex8]{biblatex}
\bibliography{bibliography}

\begin{document}

\title{Towards a Self Guided Team Based eLearning System}

\author{J.~Harty,D~.~K.~Allison,}

\maketitle

\markboth{Towards Self Guided ELearning}{}

\begin{abstract}
Millions of learners have little chance of success in terms of personal 
development globally, the schools are missing some of the teachers, they have 
very few resources and the teaching is rote-based. eLearning has been touted as 
one way to compensate for the lack of teachers and the lack of other learning
resources, however there's patchy evidence that the currently available 
resources materially help the pupils learn skills that matter. Also the 
materials may not be particularly suited to the level, or learning syles of the 
pupils. Also the materials may be very relevant to the location, language, or 
context of the school.

This position paper captures the essence of an elearning system which may help 
to address many of the concerns, and be viable in environments and contexts were
there are few trained teachers available. It supports team-based learning where 
the learners have to be able to work unaided for extended periods. The paper 
also explores whether and how differences in the skill levels of participants 
can be used to improve the learning, and whether the pairings (etc) break down 
if there are large differences between the skill levels of the participants. In 
some cases, algorithms may be able to detect common-modes of mistakes, such as 
where learners mistakenly believe 0 - any number = 0 and pair that learner with 
someone who understands the correct approach to solving this problem so that 
pupils help each other to learn and develop their skills.

It may also be suitable and relevant for situations and contexts where there are
more, and more competent teachers, and better resources and facilities 
available. The authors are interested in designing and devising a reference
implementation available as opensource software with materials available under
permissive sharing licenses such as creative commons.
\end{abstract}

\begin{IEEEkeywords}
eLearning, paired learning, digital assessment
\end{IEEEkeywords}

% Main section includes
\section{Objectives}

\begin{itemize}
\item Support partially or unguided learning.
\item Support classification of assessment failures.
\item Support tree based learning patterns.
\item Support creation of balanced learning teams.
\item Support teacher assessment of learning needs in real time.
\item Gather compelling evidence using metrics of progress
\end{itemize}




% Learning in pairs
\section{Team Based Learning}

Advantages
\begin{enumerate}
\item More gifted student can actually see improvement through reinforced 
understanding of the material.
\item In a context where the teacher may have to manage very large class sizes
this could help them focus on the most relevant needs.
\item In some situations it may also be possible for the model to be extended
outside of the classroom.
\end{enumerate}


% Team selection based on achievement level
\section{Balancing Learning Group}

One point of discussion is whether the make-up of learning groups effects their
performance of a learning task. If a collection of students were assessed on
their competence on a ten point scale, it is not clear if balancing group 
ability would enhance the performance of the learners.

While there is evidence that for some learning activities like software 
engineering pair learning has produced positive results, there is less empirical
research into using pair approaches. This seems largely due to a preference for
learning by 'rote'. 

In many deprived contexts one of the major challenges for the teacher is simply
the delivery of all of the content necessary for the curriculum. It seems
though that this style of teaching is preferred due to its simplicity; due to
the fact that lessons can be planned minute by minute in advance.

As well as the assessment of students an eLearning system could provide tools to
allow group selection based on previous performance. Assuming that all grades
of achievement could be combined the system could regularly regroup students
in order to create balanced teams. The goal here being to help the students who
are 'lagging' behind others by pairing them with other students who are 
progressing more rapidly in a particular subject. 


% Classification of assessment outputs
\section{Assessment Classification}

From interviews with education experts one of the primary roles of an effective
teacher is to identify which students are in need of support to help them 
develop their understanding of a subject. 

In some studies teachers have been tracked via video camera to analyse their 
ability to anticipate the needs of the students. This feedback is then delivered
to the teacher to help them to become more aware of their approach.

As well as detection the teacher provides a crucial role in identifying the 
learning requirement.

To support these two activities a system could, through continuous assessment
of a students understanding analyse and support two functions beyond tracking
'correct' responses:

\begin{itemize}
\item Classify the types of 'error' in the responses from a student. The 
different responses can then be used to identify learning needs.
\item Provide a real time overview for a teacher of the 'time' needed by a student to complete an assessment exercise.
\end{itemize}

The goal being to understand responses beyond the correct one as a signal which
could be used as part of a machine learning solution using a 'trained' model 
of error classification and remedial exercises.




% Motivation of the teacher
\section{Teacher Motivation}

Beyond the assumed impetus of a teacher to support their students in their 
development, in some contexts more thought may be needed into how teachers 
can be motivated.

There are enough anecdotal examples for us to generalise that large scale 
charity funded initiatives fail to live up to their potential due to limited
engagement. One cited reason for this is the need to engage 'gatekeepers' who
need significant motivation to deviate from the familiar and predictable 
incumbent approach. 

While it may not be possible to fully align an assessment system with the local
syllabus it should be apparent how any solution would aid in academic 
achievement. 

One option which was discussed was establishing a certificate around the usage and delivery of the system described above. This certificate would then, if 
recognised, aid the development of the teacher also.





\section{Related work}
Various learning packages are in widespread use, these include opensource projects such as \cite{kalite} which includes tracking progress of logged-in learners, and others such as \cite{rachel} that provide various materials using a local web server that serves opencontent such as \cite{wikipedia-for-schools} and optionally \cite{kalite}. Of the various opensource projects the one best able to support localisation is \cite{kalite}. 

The providers of \cite{kalite}, an orgainsation called \cite{learning-equality}, want to develop a new platform able to incorporate a wider range of materials and topics from many sources, rather than the pure \cite{khanacademy} contents that \cite{kalite} incorporates.

\section{Needs}
There are various needs or requirements to enable and support learning globally. These broadly include locally curated contents and materials, learning in languages in local use by the learners, and content that can be adapted to the context of the learner, for instance with local buildings, heat sources, curriculum topics, etc. 

\section{Current snapshot}
Language support is often minimal, or lags the breadth, depth or currency of the materials in the primary language, which is often US English. Translations may be partial, for instance some videos have voiceovers and/or sub-titles in another language, however the video contents are in another language, again often US English. Even when software has been fully translated the contents may include alien references e.g. to things that are commonly recognised in the USA but not in less affluent parts of the world such as a village school in Tanzania. Also contents such as geographic maps may lack the detail desired, for instance KGeography https://edu.kde.org/applications/school/kgeography/ does not include city or regional maps in India, so pupils stuggled to find relevant materials in the package.

Gaps between curriculum and content: even the larger projects such as \cite{khanacademy} are often not aligned with national curriculae. Some organisations have been working to map materials to their respective state or national curriculum, for instance numeric.org (for various years of the South African curriculum) and Cigital (for class 10 in Indian states that use the Telegu language).

\subsection*{Technical challenges}
Reliable power, connectivity, or appropriate computers. Cost of equipment and materials.

\subsection*{Other challenges}
Teacher motivation & support, particularly in state run schools in India and Kenya, and quite possibly in many other countries and states.

Logistics

Corruption

Maintenance & Support

\subsection*{Evidence}
Few of the projects have more than informal feedback that their work has been used or found useful. 

Learning practices that seem to have some evidence include group learning where peers, often with a difference in their competence in a topic, work together and the more competent helps the others learn the topic by teaching it. Examples include \cite{beautifultree}, discussions with various people including Simon Adams, University faculty who grew up in Poland, and the Pathashaala School in Tamil Nadu State.

\subsubsection*{Pathashaala School}
LearnerEducators EducatorLearners, peer-based with even the teachers considered peers of the pupils. 

\section{Outline Proposal}
Team based learning, software tracks progress, recommends pairing, and helps identify teachable-moments where a teacher can inject some additional personal help for learners. 

% Include all citations in the database
\nocite{*}


% A non numbered section
\section{Acknowledgements}
The authors wish to thank the anonymous reviewers for their valuable
suggestions.

\printbibliography

\end{document}



